\documentclass[a4paper,parskip=half]{scrartcl}
\usepackage[utf8]{inputenc}
\usepackage[T1]{fontenc}
\usepackage[english]{babel}
\usepackage{amsmath,amssymb,mathtools,geometry}
\usepackage[textsize=footnotesize]{todonotes}
\usepackage[hyperref,standard,thmmarks]{ntheorem}
\usepackage[colorlinks=true,linkcolor=blue]{hyperref}
\usepackage{cleveref,comment,xspace,alltt,subcaption,here}

\synctex=1

\renewtheorem{theorem}{Theorem}[section]
\renewtheorem{proposition}[theorem]{Proposition}
\renewtheorem{lemma}[theorem]{Lemma}
\renewtheorem{corollary}[theorem]{Corollary}
\theorembodyfont{\upshape}
\renewtheorem{example}[theorem]{Example}
\renewtheorem{remark}[theorem]{Remark}
\renewtheorem{definition}[theorem]{Definition}

\newcommand*{\iRRAM}{\texttt{iRRAM}\xspace}
\newcommand*\mto{\rightrightarrows}

\author{Franz Brauße}
\title{LLL}
\date{\today}

\begin{document}
\newgeometry{left=2mm,right=2mm,top=2.5cm,bottom=5cm}
\begin{figure}[h]
\centering
\begin{tabular}{l|l|l|l@{\;\,\vline\,}c@{\,\vline\,}c@{\,\vline\;\,}l}
  opcode & imm. & stack before & stack after & mv & $\bot$ & comment \\ \hline
  \ttfamily dup & u64:$n\;k$ & $a_1..a_n$ & $a_1..a_n\;a_1..a_k$     &&& duplicate values on the stack \\
  \ttfamily pop & u64:$n$    & $a_1..a_n$ &                          &&& pop $n$ entries off the stack \\
  \ttfamily rot & u64:$n\;k$ & $a_1..a_n$ & $a_{k+1}..a_n\;a_1..a_k$ &&& rotate $n$ entries on stack left by $k<n$ \\
  \hline
  \ttfamily tpack &         & $a_1..a_n$ i64:$n$        & $(a_1..a_n)$                  &&& pack $n$ elements from stack into a new tuple \\
  \ttfamily texpl &         & $(a_1..a_n)$              & $a_1..a_n$ i64:$n$            &&& unpack tuple onto stack \\
  \ttfamily tsplt &         & $(a_1..a_n)$ i64:$k$      & $(a_1..a_k)$ $(a_{k+1}..a_n)$ &&& split tuple at index $k\in\{0,\ldots,n\}$ \\
  \ttfamily tcat  &         & $(a_1..a_n)$ $(b_1..b_m)$ & $(a_1..a_n,b_1..b_m)$         &&& concatenate tuples \\
  \ttfamily tlen  &         & $(a_1..a_n)$              & $(a_1..a_n)$ i64:$n$          &&& extract length of tuple \\
  \hline
  \ttfamily apush & adr:$f$ &                    & $f$        &&& push immediate address to stack \\
  \ttfamily dcall &         & adr:$f$ $a_1..a_n$ & $b_1..b_m$ &&& call dyn. $f(a_1,\ldots,a_n) = (b_1,\ldots,b_m)$, $n,m$ unspec. \\
  \ttfamily scall & adr:$f$ & $a_1..a_n$         & $b_1..b_m$ &&& call imm. $f(a_1,\ldots,a_n) = (b_1,\ldots,b_m)$, $n,m$ unspec. \\
  \ttfamily ret   &         &                    &            &&& return from function call or exit \\
  \ttfamily jmp   & adr:$l$ &                    &            &&& unconditional jump to $l$ \\
  \ttfamily jnz   & adr:$l$ & i64:$v$            & $v$        &&& conditional jump to $l$ if $v\neq 0$ \\
  \hline
  \ttfamily or    &         & i64:$v$ i64:$w$ & i64:$x\in\{0,1\}$ &&& boolean disjunction $x=0\iff v=0=w$ \\
  \ttfamily and   &         & i64:$v$ i64:$w$ & i64:$x\in\{0,1\}$ &&& boolean conjunction $x=1\iff v\neq0\neq w$ \\
  \ttfamily not   &         & i64:$v$         & i64:$x\in\{0,1\}$ &&& boolean negation $x=1\iff v=0$ \\
  \hline
  \ttfamily ipush & i64:$v$ &                 & i64:$v$          && & push imm.\ 64-bit two's complement $v$ to stack \\
  \ttfamily ineg  &         & i64:$v$         & i64:$(-v)$       && & 64-bit two's complement negation \\
  \ttfamily iadd  &         & i64:$v$ i64:$w$ & i64:$(v+w)$      && & 64-bit binary addition with overflow \\
  \ttfamily imul  &         & i64:$v$ i64:$w$ & i64:$(v\cdot w)$ && & 64-bit two's complement multiplication w/ overflow \\
  \ttfamily idiv  &         & i64:$v$ i64:$w$ & i64:$(v/w)$      &&y& 64-bit two's complement division with truncation \\
  \ttfamily isgn  &         & i64:$v$         & i64:$(\operatorname{sgn}v)$ &&& sign of 64-bit two's complement int \\
  \hline
  \ttfamily zconv &         & i64:$v$         & Z:$v$            && & convert 64-bit two's complement int to integer \\
  \ttfamily zneg  &         & Z:$v$           & Z:$(-v)$         && & integer negation \\
  \ttfamily zadd  &         & Z:$v$ Z:$w$     & Z:$(v+w)$        && & integer addition \\
  \ttfamily zmul  &         & Z:$v$ Z:$w$     & Z:$(v\cdot w)$   && & integer multiplication \\
  \ttfamily zdiv  &         & Z:$v$ Z:$w$     & Z:$(v/w)$        &&y& integer division with truncation \\
  \ttfamily zsgn  &         & Z:$v$           & i64:$(\operatorname{sgn}v)$ &&& sign of integer \\
  \ttfamily zsh   &         & Z:$v$ i64:$n$   & Z:$(v\cdot2^n)$  && & integer multiplication by $2^n$ with truncation \\
  \hline
  \ttfamily rconv &         & Z:$v$           & R:$v$           & && convert an integer to real \\
  \ttfamily rneg  &         & R:$v$           & R:$(-v)$        & && real negation \\
  \ttfamily radd  &         & R:$v$ R:$w$     & R:$(v+w)$       & && real addition \\
  \ttfamily rinv  &         & R:$v$           & R:$(1/v)$       & &y& real inversion \\
  \ttfamily rmul  &         & R:$v$ R:$w$     & R:$(v\cdot w)$  & && real multiplication \\
  \ttfamily rsh   &         & R:$v$ i64:$w$   & R:$(v\cdot2^w)$ & && multiplication by $2^w$ \\
  \ttfamily rin   &         &                 & $(a_1..a_n)$    & && get real input \\
  \hline
  \ttfamily rlim  & adr:$f$ & $a_1..a_n$ & $b_0..b_m$ &?&y& $\lim_{p\to\infty}{(f(-p,a_1..a_n))_p} = (b_0..b_m)$, $n,m$ unspec. \\[-1.35ex]\hline\noalign{\vspace{\dimexpr 1.35ex-.5pt}}
  \ttfamily rch   &         & R:$r_1..r_n$ i64:$n$ & i64:$k$    &y&& mv-choice, $\{0\}$ if all $<0$, $\{i:r_i>0\}$ otherwise \\
  \ttfamily rapx  &         & R:$x$ i64:$p$   & Z:$m$ i64:$e$   &y&& approx.\ to abs.\ prec.: $|x-m\cdot2^{e-\lceil\log_2(|m|+1)\rceil}|<2^p$ \\
  \ttfamily rilog &         & R:$x$ i64:$p$   & i64:$k$         &y&& approx.\ integer logarithm $\{k:2^k\leq|x|+2^p<2^{k+2}\}$ \\
  \hline
  \ttfamily entc  &         &                           & Z:$\tilde p$ && & enter continuous section with (volatile) prec.\ $\tilde p$ \\
  \ttfamily lvc   & u64:$n$ & Z:$\tilde p$ $a_1'..a_n'$ & $a_1..a_n$   &&y& leave continuous section (last $\implies\tau(a_i)\neq\text{R}$)
\end{tabular}
\caption{Instruction set of the low-level language.}
\label{fig:instrs}
\end{figure}

\restoregeometry
\newpage

\newcommand*\dom{\operatorname{dom}}
\newcommand*\Top{\operatorname{top}}

Let $\tau\coloneqq\{\text{i64},\text{adr},\text Z,\text K,\text R\}$ and for
$t\in\tau$ let
\[ \dom t\coloneqq\begin{cases}
	\{-2^{63},\ldots,2^{63}-1\}\subseteq\mathbb Z &\text{if}~t=\text{i64} \\
	\mathbb Z_{2^{64}} &\text{if}~t=\text{adr} \\
	\mathbb Z &\text{if}~t=\text Z \\
	\mathbb K &\text{if}~t=\text K \\
	\mathbb R &\text{if}~t=\text R
\end{cases} \]
and let $\Top t$ be the discrete topology on $\dom t$ for $t\in\{\text{i64},\text{adr},\text Z\}$,
the topology $\{\varnothing,\{0\},\{1\},\{0,1\},\{0,1,\bot\}\}$ of the lifted
booleans $\dom t$ for $t=\text K$ and the standard topology on the real line
$\dom t$ for $t=\text R$. Note that for all $t\in\tau$, $(\dom t,\Top t)$ are
complete and for $t\neq\text K$ these are also metric spaces. In the following
$d:(\dom t)^2\to\mathbb R$ denotes the respective metric if it exists.
For finite sequences $s\in\tau^*$ let
$\dom s\coloneqq\bigtimes_{i=1}^{|s|}\dom s_i$
be the product space of $(\dom s_i)_i$, $\Top s$ be its product topology and
if all $\dom s_i$ are metric spaces, let also $d:(\dom s)^2\to\mathbb R$ denote
the metric induced by $\Vert\cdot\Vert_\infty$ on $\dom s$.


Let $s\in(\tau\setminus\text K)^*$ and $\tilde p\in\mathbb Z$ and $x,x'\in\dom s$.
Then $x'$ is a \emph{$\tilde p$-approximation} of $x$ if
$d(x,x')\leq2^{\tilde p}$.

Let $\mathcal T=\bigcup_{t\in\tau}\dom t$ and let $p$ be a \emph{program}, that
is, a finite word over the set of
instructions from \cref{fig:instrs} of length $n\leq 2^{64}$.
We call $(c,v,s,r)$
a \emph{configuration of $p$} where $c\in\mathbb Z_{2^{64}}$ with $c<n$ is the
\emph{program counter}, $v\in\mathcal T^*$ is the
\emph{value stack}, $s\in(\mathbb Z_{2^{64}})^*$ is the \emph{continuous section
stack} and $r\in(\mathbb Z_{2^{64}})^*$ is the
\emph{return stack}.
For any \emph{program} $p$ (that is, ) of length $n$ and any $1\leq i\leq n$,
$(i,\epsilon,\epsilon,\epsilon)$ is an \emph{initial configuration of $p$},
where $\epsilon$ denotes the empty word.

We will now give the context in which \cref{fig:instrs} defines the transition
relation $\vdash$ on configurations of $p$ for a program $p$. Let $(c,v,s,r)$ be
a configuration of $p$ and let $I=p_c$.
Then $(c,v,s,r)\vdash(c',v',s',r')$ iff
\begin{itemize}
\item
\end{itemize}


%With the exception of $\texttt{entc},\texttt{lvc}$ and $\texttt{dcall},\texttt{scall},


If code inside a continuous section enclosed by the pair of instructions
$(\texttt{entc},\texttt{lvc}~n)$ computes a $\tilde p$-approximation
$(a_1',\ldots,a_n')$ of $(a_1,\ldots,a_n)$, then the continous section computes
$(a_1,\ldots,a_n)$.

\newpage

How to implement \verb!main()! depends on what we want to express by our
program. Should it compute $f:\mathbb R\to\mathbb R$,
$g:\mathbb Z\times\mathbb R\to\mathbb Q$ or
$h:\mathbb Z\times\mathbb Q\to\mathbb Q$?

\begin{figure}[h]
\small
\begin{minipage}[b]{.33\linewidth}
\begin{verbatim}
#include <iRRAM/lib.h>
/* define input() and
 * output() via kirk */

using namespace iRRAM;
int main() {
  iRRAM_init();
  exec([]{

    REAL x = input();
    REAL y = sqrt(x);
    output(y);

  }); }
\end{verbatim}
\subcaption{$f:\mathbb R\to\mathbb R$}
\end{minipage}%
\begin{minipage}[b]{.33\linewidth}
\begin{verbatim}
#include <iRRAM/lib.h>
/* define input()
   using kirk */

using namespace iRRAM;
int main() {
  iRRAM_init();
  exec([]{
    int n; cin >> n;
    REAL x = input();
    REAL y = sqrt(x);
    cout << setRwidth(n)
         << y;
  }); }
\end{verbatim}
\subcaption{$g:\mathbb Z\times\mathbb R\to\mathbb Q$}
\end{minipage}%
\begin{minipage}[b]{.33\linewidth}
\begin{verbatim}
#include <iRRAM/lib.h>



using namespace iRRAM;
int main() {
  iRRAM_init();
  exec([]{
    int  n; cin >> n;
    REAL x; cin >> x;
    REAL y = sqrt(x);
    cout << setRwidth(n)
         << y;
  }); }
\end{verbatim}
\subcaption{$h:\mathbb Z\times\mathbb Q\to\mathbb Q$}
\end{minipage}
\caption{Implementations of square root with different composeability.}
\end{figure}
The only line common to the continous parts of the algorithms is
\verb!REAL y = sqrt(x);!, which is exactly the composeable part of these
algorithms.

Actually, the kirk-versions are cheated, it might look like \cref{fig:kirk-lib}.
\begin{figure}[H]
\begin{verbatim}
#include <kirk/kirk-irram.hh>

extern "C" void sqrt(kirk_real_t **in, int n_in, kirk_real_t **out) {
  iRRAM_init();
  using namespace iRRAM;
  assert(n_in == 1);
  auto machine = kirk::irram::eval(in, n_in, out, 1,
    [](const REAL *in, REAL *out){
      const REAL &x = in[0];
      REAL &y = out[0];
      y = iRRAM::sqrt(x);
    });
  /* can't make use of the machine, yet, forget it */
}

/* what should main() do? */
\end{verbatim}
\caption{Library-like implementation of $f:\mathbb R\to\mathbb R$.}
\label{fig:kirk-lib}
\end{figure}

What should \verb!main()! do? The point being, in the discrete setting of
\verb!main()!, ``executing'' a function on continuous data does not make sense
using a model like oracle machines. Only as a transformation of a stream of
approximations. Therefore, there are two options.
\begin{enumerate}
\item
	Implement algorithms on continuous data not in terms of \verb!main()!
	but as library functions that operate on (e.g.\ kirk-provided) function
	pointers as in \cref{fig:kirk-lib}.
\item
	Transform a stream of approximations from \verb!stdin! to \verb!stdout!,
	an example is provided in \cref{fig:stream}.
\end{enumerate}

\begin{figure}[H]
\begin{verbatim}
#include <kirk/kirk-c-types.h>
int main() {
  kirk_real_t *x[] = { kirk_real_from_file(stdin) };
  kirk_real_t *y[1];
  sqrt(x, 1, y);
  kirk_real_to_file(y[0], stdout); /* returns only when stream errors */
}
\end{verbatim}
\caption{Stream-like implementation of $f:\mathbb R\to\mathbb R$.}
\label{fig:stream}
\end{figure}

It does not seem as if a program like \cref{fig:stream} in general would be of
much use.

With respect to composeability, it is my impression that a design like
$g:\mathbb Z\times\mathbb R\to\mathbb Q$ or
$h:\mathbb Z\times\mathbb Q\to\mathbb Q$ is not the right choice for the
language. Therefore, programs in this language are meant to be library-like,
i.e.\ for the stack-based variant of the low-level language this would mean
an initial configuration where there already are real numbers (type R) on the
stack and when the program returns, it leaves zero or more objects of type R on
the stack.

Programs that expect this kind of input/output have to be executed in a
continuous section or a limit respectively and they are library-like functions,
that is, not \verb!main()!.

\end{document}
